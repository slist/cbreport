\usepackage{tikz}
\usepackage{enumitem}
\usepackage{setspace}
\usepackage{times}
\usepackage{tcolorbox}
\usepackage{listings}
\usepackage{color}
\usepackage{awesomebox}
\usepackage{float} %To be able to use H on figure to force position. See https://tex.stackexchange.com/questions/16207/image-from-includegraphics-showing-up-in-wrong-location

\usepackage[toc,section=section]{glossaries}
\newglossaryentry{ns}
{
	name={namespaces},
	description={In Kubernetes, namespaces provides a mechanism for isolating groups of resources within a single cluster. Names of resources need to be unique within a namespace, but not across namespaces.}
}

\newglossaryentry{cndr}
{
	name={CNDR},
	description={Cloud Native Detection and Response}
}

\newglossaryentry{active}
{
	name={active},
	description={The sensor is periodically performing a check-In to the VMware Carbon Black Cloud console. If the sensor could do it within the last 30 days, then the sensor is showing as Active. This does not mean that the Device is turned on or reachable.}
}

\newglossaryentry{enforcement}
{
	name={enforcement},
	description={Instead of blocking Carbon Black Container can mutate a mutate a deployment to add security features}
}

\makeglossaries

\usepackage[colorlinks = true,
linkcolor = blue,
urlcolor  = blue,
citecolor = blue,
anchorcolor = blue]{hyperref}

\usepackage{geometry}
\geometry{hmargin=1.5cm,vmargin=1.5cm}

\usepackage{pgfplots}
\usepackage{pgfplotstable}

%
% These colour codes are taken from the RGB swatches in Brand Central.
% https://www.vmware.com/brand/portal/guidelines/color.html
%
\definecolor{vmwBlack}{HTML}{000000}
\definecolor{vmwWhite}{HTML}{ffffff}
\definecolor{vmwGrey}{HTML}{717074}

% Primary colours
\definecolor{vmwOcean}{HTML}{0091DA}
\definecolor{vmwLeaf}{HTML}{78BE20}
\definecolor{vmwIndigo}{HTML}{1D428A}

% Secondary colours; never use alone.
\definecolor{vmwPlum}{HTML}{7F35B2}
\definecolor{vmwAqua}{HTML}{00C1D5}

\definecolor{darkblue}{HTML}{0070E0}

% Use VMware font : Metropolis
\usepackage{fontspec}
\setmainfont[Ligatures=TeX]{Metropolis Light} % default font, requires fontspec package
\newfontfamily\vmwFontMetropolis{Metropolis}
\newfontfamily\vmwFontMetropolisLight{Metropolis Light}
%--> Need to use XeLaTex

%
% Table formatting
%
\usepackage{tabularray}

%
% Table styles
%

% This environment replicates the "vmw_table heading left" style defined in the VMware Microsoft Office template.
\newcommand{\styleVmwTableHeadingLeft}{
	\fontMetropolis
	\fontsize{10pt}{10pt}\selectfont % font size, line spacing
	\color{vmwBlack}
}

% This macro replicates the "vmw_table heading top" style defined in the VMware Microsoft Office template.
\newcommand{\styleVmwTableHeadingTop}{
	\fontMetropolis
	%  \fontsize{9.5pt}{9.5pt}\selectfont % font size, line spacing
	\fontsize{12pt}{12pt}\selectfont % font size, line spacing
	\color{vmwWhite}
	\bfseries
}

% This macro replicates the "vmw_table body copy" style defined in the VMware Microsoft Office template.
\newcommand{\styleVmwTableBodyCopy}{
	\fontsize{8pt}{8pt}\selectfont % font size, line spacing
	\color{vmwBlack}
}

\setlength{\arrayrulewidth}{0.6pt} % thickness of lines in tables

\NewTblrEnviron{vmwTable}
\SetTblrInner[vmwTable]{
	hlines = {vmwGrey},
	vlines = {vmwGrey},
	rowsep = 5pt,
	% The order of the settings below is important; in case of a conflict, the later settings take precedence.
	row{odd} = {bg=vmwGrey!20},
	row{even} = {bg=white},
	row{2-Z} = {font=\styleVmwTableBodyCopy},
	column{1} = {font=\styleVmwTableHeadingLeft},
	row{1} = {bg=accent1,font=\styleVmwTableHeadingTop},
}
\SetTblrOuter[vmwTable]{baseline=B}

\NewTblrEnviron{vmwTablePlain}
\SetTblrInner[vmwTablePlain]{
	hline{2-Z} = {vmwGrey},
	rowsep = 5pt,
	column{1} = {font=\styleVmwTableHeadingLeft},
	column{2-Z} = {font=\styleVmwTableBodyCopy},
}
\SetTblrOuter[vmwTablePlain]{baseline=B}

\newcommand{\nicecounter}[2]{
	\begin{tikzpicture}[
		e/.style={rounded corners, fill=vmwOcean, font={\sffamily\bfseries\Huge}, text=white, align=center, minimum width=130pt, text height=22pt},
		f/.style={rounded corners, fill=vmwLeaf, font={\sffamily\bfseries\large}, text=white, align=center, minimum width=130pt, text height=15pt},
		]
		\node(c)[e]{#1};
		\node(d)[f]at(c.south)[yshift=-12pt]{#2};
	\end{tikzpicture}
}

\newcommand{\cbinstance}[1]{
	\begin{center}
		\large{#1}
	\end{center}
	\vspace{0.5cm}
}

\newcommand{\department}[1]{
	\begin{center}
		\large{#1}
	\end{center}
}

\renewcommand{\author}[1]{
	\begin{center}
		\Large{#1}
	\end{center}
	\vspace{0.5cm}
}
\renewcommand{\title}[1]{
	\vspace{3cm}
	\begin{center}
		\huge{#1}
	\end{center}
	\vspace{1.7cm}
}

\renewcommand{\date}[2]{
	\begin{center}
		\normalsize{#1 #2}
	\end{center}
	\vspace{0.5cm}
}

\newcommand{\bannercbcontainer}[0]{
	\begin{tikzpicture}[remember picture,overlay]
		\node[anchor=north west, inner sep=0pt] at (current page.north west){\includegraphics[width=\paperwidth]{cbbanner.jpg}};
	\end{tikzpicture}
}

\newcommand{\logocbcontainer}[0]{
	\begin{center}
		\includegraphics[width=4.5cm]{cb.png}
	\end{center}
}

\usepackage{fontsize}
\changefontsize[14pt]{11pt}

\usepackage{listings}
\lstset{
	basicstyle={\ttfamily\footnotesize},
}

\makeatletter
\newcommand\verbfile[1]{
	\begingroup
	\let\do\@makeother\dospecials
	\obeyspaces\obeylines\ttfamily
	\input#1\relax
	\endgroup
}
\makeatother

\parindent=0cm
