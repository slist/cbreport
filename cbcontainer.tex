\section{Introduction}

This is a sample report created from internal demo Kubernetes clusters.

\vskip15pt
\par

For more information about VMware products vulnerabilities, please visit:

\href{https://www.vmware.com/security/advisories.html}{VMware Security Advisories}


\section{Kubernetes clusters}

This is the list of Kubernetes clusters protected by VMware Carbon Black Container.
\vskip15pt

\verbfile{./cluster_list.txt}

\section{Namespaces}

This is the list of \gls{ns} protected by VMware Carbon Black Container.
\vskip15pt

\verbfile{./namespace_list.txt}

\section{Container Registries}
\subsection{Container Registries detected}
This is the list of container Registries detected by VMware Carbon Black Container.
\vskip15pt

\verbfile{./registry_list.txt}

\vskip15pt

"docker.io" aka "Docker Hub" is the default container registry for Docker and Kubernetes.

\subsection{About Docker Hub}
\subsubsection{Docker Hub Rate limit}
\href{https://tanzu.vmware.com/developer/guides/harbor-as-docker-proxy/}{How to use Harbor Registry to Eliminate Docker Hub Rate Limits}

\subsubsection{Docker Hub malicious container images}

\begin{quote}
Over 1,600 publicly available Docker Hub images hide malicious behavior, including cryptocurrency miners, embedded secrets that can be used as backdoors, DNS hijackers, and website redirectors.
\end{quote}
\href{https://www.bleepingcomputer.com/news/security/docker-hub-repositories-hide-over-1-650-malicious-containers/}{source: Bleeping Computer}

\begin{tipblock}
Carbon Black Container can block Docker Hub usage. Some free open source container registries could be used, for example \href{https://goharbor.io/}{Harbor}.
\end{tipblock}

\section{Hardening Policies}
%\noindent

This section list all security policies, and provides the number of exceptions and violations.

\begin{table}[h!]
	\centering
	\begin{tabular}{ |c|c|c|c| }
		\hline
		Policy name & Scope & Exceptions & Violations \\ 
		\hline\hline
		system & system & 0 & 6\\
\hline
demo & Any & 3 & 4\\
\hline
\end{tabular}

%	\end{tabular} --> included in hardeningpolicies
\caption{Hardening policies.}
\end{table}

\begin{importantblock}
	Violations can be: alert or block or \gls{enforcement}. Modify your infrastucture to be compliant, and last option can be to add some exceptions.
\end{importantblock}

\section{Security alerts}
\noindent

This is a summary of alerts reasons, for more details, please see VMware Carbon Black console:
\vskip10pt

US: \href{https://defense-prod05.conferdeploy.net/}{https://defense-prod05.conferdeploy.net/}

EU: \href{https://defense-eu.conferdeploy.net/}{https://defense-eu.conferdeploy.net/}

NRT: \href{https://defense-prodnrt.conferdeploy.net/}{https://defense-prodnrt.conferdeploy.net/}

\vskip10pt

\begin{obeylines}
\input{./alerts_reason.txt}
\end{obeylines}


\subsection{Alerts - Hardening - K8s Events}

%Container running as \textbf{root} detected in your Kubernetes clusters.

\subsection{Alerts - Network}
\noindent

%\textbf{Port scan} detected in your Kubernetes clusters.

Connection to \textbf{malicious IP} :

%\href{malware.com}{malware.com}

\verbfile{./alerts_remoteip.txt}

\begin{importantblock}
If your Kubernetes cluster connects to a malicious IP, investigate! 
\end{importantblock} 

\subsection{Alerts - CNDR}

CNDR = Cloud Native Detection and Response

\paragraph{}

\textbf{CNDR} alerts are:
\begin{itemize}
	\item CB analytics (NGAV - Next Gen Anti Virus)
	\item Watchlists (EEDR - Enterprise Endpoint Detection and Response)
\end{itemize}

xmrig detected in your Kubernetes clusters.

\section{Malware and Secret Detection}
\subsection{Malware}
\noindent

TODO\\
There is 0 malware in your Kubernetes clusters.

\subsection{Secret Detection}
\noindent

TODO\\
There is 0 secret in your Kubernetes clusters.

\section{Critical vulnerabilities}

Only container images with critical vulnerabilities will be listed below.

%Vulnerabilities color code
\definecolor{darkred}{RGB}{106,2,12}
\definecolor{red}{RGB}{211,18,18}
\definecolor{orange}{RGB}{253,136,36}
\definecolor{yellow}{RGB}{255,202,58}
\definecolor{lightgrey}{RGB}{187,199,204}

\newtcbox{\criticalvulncbox}{colframe=darkred,colback=darkred,top=0pt,bottom=0pt,colupper=white}
\newtcbox{\highvulncbox}{colframe=red,colback=red,top=0pt,bottom=0pt,colupper=white}
\newtcbox{\mediumvulncbox}{colframe=orange,colback=orange,top=0pt,bottom=0pt,colupper=white}
\newtcbox{\lowvulncbox}{colframe=yellow,colback=yellow,top=0pt,bottom=0pt}
\newtcbox{\unknownvulncbox}{colframe=lightgrey,colback=lightgrey,top=0pt,bottom=0pt,colupper=white}

\vskip10pt

\tcbset{on line}
Color code:
\criticalvulncbox{Critical}
\highvulncbox{High}
\mediumvulncbox{Medium} 
\lowvulncbox{Low} 
\unknownvulncbox{Unknown} 

Example: 
\criticalvulncbox{200/100} = 
200 critical vulnerabilities / 100 critical vulnerabilities with fix available
\par

\begin{center}
\rule{0.5\linewidth}{1pt}
\end{center}


%--------------------------
% Critical
%--------------------------
\subsection{default:dev}

gke.gcr.io
metrics-server

\criticalvulncbox{1/1}
\highvulncbox{8/8}
\mediumvulncbox{4/4}
\lowvulncbox{2/2}

\subsection{default:prod}

docker.io
vmwareallspark/acme-load-gen
latest
\criticalvulncbox{213/188}
\highvulncbox{1461/1198}
\mediumvulncbox{1794/1443}
\lowvulncbox{3434/2346}


gcr.io
vmwarecloudadvocacy/acmeshop-cart
1.0.0
\criticalvulncbox{62/7}
\highvulncbox{210/27}
\mediumvulncbox{187/27}
\lowvulncbox{10/4}


gcr.io
vmwarecloudadvocacy/acmeshop-order
1.0.1
\criticalvulncbox{56/7}
\highvulncbox{207/32}
\mediumvulncbox{184/29}
\lowvulncbox{7/5}


docker.io
tamirmich/log4j2-demo
0.0.3
\criticalvulncbox{46/6}
\highvulncbox{116/43}
\mediumvulncbox{84/20}
\lowvulncbox{6/3}


gcr.io
vmwarecloudadvocacy/acmeshop-front-end
rel1
\criticalvulncbox{38/4}
\highvulncbox{218/36}
\mediumvulncbox{173/31}
\lowvulncbox{11/9}


docker.io
weaveworksdemos/catalogue-db
0.3.0
\criticalvulncbox{30/26}
\highvulncbox{106/83}
\mediumvulncbox{113/74}
\lowvulncbox{7/5}


gcr.io
vmwarecloudadvocacy/acmeshop-payment
1.0.0
\criticalvulncbox{24/4}
\highvulncbox{133/29}
\mediumvulncbox{91/26}
\lowvulncbox{11/9}


docker.io
library/rabbitmq
3.6.8-management
\criticalvulncbox{23/19}
\highvulncbox{98/74}
\mediumvulncbox{152/78}
\lowvulncbox{9/6}


docker.io
weaveworksdemos/user-db
0.3.0
\criticalvulncbox{22/20}
\highvulncbox{103/80}
\mediumvulncbox{124/84}
\lowvulncbox{8/6}


gcr.io
vmwarecloudadvocacy/acmeshop-user
1.0.0
\criticalvulncbox{21/11}
\highvulncbox{81/23}
\mediumvulncbox{57/17}
\lowvulncbox{7/5}


docker.io
library/redis
5.0.3-alpine
\criticalvulncbox{14/2}
\highvulncbox{74/8}
\mediumvulncbox{48/14}
\lowvulncbox{7/4}


gcr.io
vmwarecloudadvocacy/acmeshop-catalog
rel1
\criticalvulncbox{14/4}
\highvulncbox{73/15}
\mediumvulncbox{56/16}
\lowvulncbox{6/4}


docker.io
openso/olivetin
latest
\criticalvulncbox{2/0}
\highvulncbox{16/4}
\mediumvulncbox{17/3}


docker.io
library/redis
alpine
\criticalvulncbox{2/0}
\highvulncbox{6/0}
\mediumvulncbox{6/0}
\lowvulncbox{1/0}



